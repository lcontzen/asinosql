\documentclass[11pt]{article}
\usepackage[utf8]{inputenc}
\usepackage[french]{babel}
\usepackage[T1]{fontenc}
\usepackage{verbatim}
\usepackage{graphicx}
\usepackage{fullpage}
\usepackage{hyperref}
\author{Contzen Laurent}

\begin{document}

\section{Slides titre + table des matières}
Ohai! Mon sujet est donc la mouvance NoSQL dans le monde des gestionnaires de bases de données. NoSQL est à interpreter, comme vous allez le voir, comme Not only SQL et non Not SQL. <Description table des matières>
\section{Historique}
Les bases de données sont d'une importance capitale depuis les débuts de l'informatique dont un des buts était de pouvoir stocker et manipuler beaucoup de données facilement. Au début, chaque système était lié à son gestionnaire de bases de données, ce n'était ni optimisé ni pratique et on a vite ressenti le besoin de normaliser tout ca. Une première standardisation est arrivée en 1971, la Codasyl Approach qui était navigationnelle. C'est plus tard dans la même décénie que le modèle relationnel est apparu suite à un article d'un ingénieur de chez IBM et le langage de requêtes SQL suivi très rapidement. Ce modèle est resté sans réelle concurrence pendant plusieurs décénies jusqu'à récemment lorsqu'en 2009 la mouvance NoSQL s'est lancée suite aux besoins des réseaux sociaux.
\section{Le modèle relationnel}
Pour bien comprendre cette mouvance qui veut se différencier de ce modèle historique il est nécéssaire de connaitre un minimum ce dernier. Contenu des slides. Exemples. slides.
\section{ACID et CAP}
Le monde des gestionnaires de bases de données comporte deux ensembles de principes. ACID. CAP.


\end{document}

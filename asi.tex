\documentclass[11pt]{article}
\usepackage[utf8]{inputenc}
\usepackage[french]{babel}
\usepackage[T1]{fontenc}
\usepackage{verbatim}
\usepackage{graphicx}
\usepackage{fullpage}
\author{Contzen Laurent}

\begin{document}

\begin{titlepage}  
  \begin{flushleft}
    Contzen Laurent
  \end{flushleft}
  \begin{center}
    \vspace{85mm}\LARGE{\textbf{STIC-B-415 : Architecture des Systèmes d'Information.} \\    
      Les bases de données de type NoSQL.}
  \end{center}
  \begin{flushright}
    \vspace{95mm}
    Année Académique 2011-2012.             
  \end{flushright}
\end{titlepage}

\tableofcontents
\newpage
\vspace*{\fill}
\begin{flushright}
  A DBA walks into a bar and leaves immediatly. \\
  He couldn't find a table. \\
  \textit{- Anonymous}
\end{flushright}
\vspace*{\fill}
\newpage

\section{Introduction}
Opposées aux bases de données relationnelles, les bases de données de faisant partie de la mouvance NoSQL (Not only SQL) sont non relationnelles, distribuées, horizontalement scalable (FIXME) et open-source. Bien que le terme NoSQL ait été utilisé pour la première fois en 1998 par Carlo Strozzi pour nommer le DBMS \footnote{Database management system} relationnel qu'il venait de développer, la vraie naissance de cette mouvance eut lieu en 2009 lorsque Eric Evans utilisa ce terme dans une discussion sur les bases de données distribuées. Ces dernières devenant de plus en plus importantes dans le web actuel, il était logique de travailler sur le sujet quitte à remettre en question un modèle en place depuis plusieures décénies.
\section{Historique}
Les bases de données ont toujours été quelque chose de fondamental en informatique. En effet, stocker de grandes quantité d'informations et pouvoir y accéder et les traiter rapidement et efficacement est un des buts fondamentaux de cette discipline. \\
Au départ, chaque système était lié à sa propre base de données, étudiée et optimisée pour lui. Ceci posait évidemment, comme beaucoup de choses à cette époque, de gros problèmes de portabilité de données et changer un système informatique avait d'énormes implications. Il était donc nécéssaire et urgent de standardiser ça, chose qui a été faite par le Database Task Group qui proposa en 1971 un standard qui fut connu en tant que ``Codasyl Approach''. Cette approche était navigationnelle et basée sur une navigation manuelle d'un ensemble d'informations disposées en ``réseau''. \\
C'est ensuite dans les années 1970 qu'un ingénieur travaillant chez IBM, Edgar Codd, révolutionna le monde des bases de données avec son article ``A Relational Model of Data for Large Shared Data Banks'' qui posa les fondations des bases de données relationnelles. Les principes énoncés dans cet articles furent implémentés, testés, et un langage de requêtes standardisé, le SQL, fit son apparition en même temps. \\
Les bases de données relationnelles utilisant SQL se sont très vite répandues et plusieures implémentations apparurent telles que DB2, Microsoft SQL Server, PostgreSQL, MySQL, etc. \\
C'est ensuite bien plus récemment, vers la fin des années 2000, que ce modèle a été remis en question et que plusieurs modèles de bases de données non relationnelles sont apparus. 
\section{Différences fondamentales avec les bases de données relationnelles}

\section{Les grandes familles}
\subsection{Les bases de données de type Key-Value}

\subsection{Les bases de données de orientées document}

\subsection{Les bases de données orientées colonnes}

\subsection{Les bases de données orientées graphe}

\subsection{Les autres bases de données}

\section{Conclusion}


\end{document}
